\begin{table}[H]
\centering
\caption{IRR Sensitivity Analysis: Synergy Scenarios}
\label{tab:irr_sensitivity}
\footnotesize
\setlength{\tabcolsep}{4pt}
\renewcommand{\arraystretch}{1.0}
\resizebox{\textwidth}{!}{%
\begin{tabular}{lrrrr}
\toprule
Scenario & Annual Synergies & IRR & NPV @ 10\% & Tx Value for \\
& (\$M) & (\%) & (\$B) & 10\% IRR (\$B) \\
\midrule
Pessimistic (50\%) & \$61.6M & -20.3\% & \$-10.70B & \$1.50B \\
Base Case (75\%) & \$92.4M & -16.4\% & \$-10.14B & \$2.05B \\
Base (100\%) & \$123.2M & -13.3\% & \$-9.58B & \$2.63B \\
Optimistic (125\%) & \$154.0M & -10.7\% & \$-9.02B & \$3.18B \\
Bull Case (150\%) & \$184.8M & -8.5\% & \$-8.47B & \$3.73B \\
Very Bull (200\%) & \$246.4M & -4.8\% & \$-7.35B & \$4.83B \\
\bottomrule
\end{tabular}%
}
\begin{flushleft}
\scriptsize
\textit{Note: Assumes 7-year holding period, 20x terminal EBITDA multiple, 3\% base FCF growth, 5\% synergy growth. Transaction value: \$12.2B. Negative IRRs indicate strategic acquisition where financial returns are secondary to strategic value. "Tx Value for 10\% IRR" shows the transaction value needed to achieve 10\% IRR at each synergy level.}
\end{flushleft}
\end{table}
